\documentclass[11pt]{article}

\usepackage[utf8]{inputenc}
\usepackage[x11names]{xcolor}   % Accès à une table de 317 couleurs
\usepackage{graphicx} %Pur utiliser la colorbox
\usepackage{textcomp}
\usepackage{amsmath}
\usepackage{amssymb}
\usepackage[ruled]{algorithm2e}

\title{\textbf{PSTL \\ Visualisation de Graphes Séries-Parallèles}}
\author{GRUENPETER Morane, RETAIL Tanguy}
\date{22/01/2016}
\begin{document}

\maketitle
\tableofcontents
\newpage


%%%%%%%%%%%%%%%%%%%%%%%%%%%%%%%%%%%%%%%%%%%%%%%%%%%%%%%%%%%%%%%%%%%%%%%%%%%%%%%%%%%%%%%%%%%%%%%%%%%%
\section{Introduction}
\fcolorbox{black}{LightBlue2}{\parbox{\linewidth}{
TODO :
\newline
Ici, on a copié/collé le sujet de la page du PSTL. Il faut faire notre introduction en gardant le sujet exact et en parlant de :
\begin{itemize}
\item En quoi ça consiste (i.e. Quoi?)
\item Qu'est-ce qui nous a motivé à faire ça (i.e. Pourquoi?)
\item Le plan et comment on va présenter notre solution (i.e. Comment?)
\end{itemize}
}}

Les récents progrès de la génération aléatoire permettent de nombreuses applications, notamment en algorithmique expérimentale (simulations), en physique, en bio-informatique ou encore au niveau du génie logiciel. Les générateurs de structures combinatoires nous permettent de générer uniformément des graphes (par exemple des arbres ou des graphes séries-parallèles) de plusieurs millions de noeuds en à peine quelques secondes. Habituellement, ces graphes sont stockés dans des fichiers textes structurés (dot, xml, ...). On veut naturellement obtenir une visualisation graphique de tels objets.
\newline
Il existe plusieurs outils permettant de visualiser des graphes quelconques en utilisant des algorithmes souvent coûteux, le plus connu étant graphviz. Dans le cas des arbres, il existe un \textbf{algorithme de visualisation en temps linéaire} (en nombre de noeuds) qui a été implémenté dans deux outils, l’un en Python (lors d’un précédent PSTL) et l’autre en Ocaml. Nous aimerions \textbf{adapter cet algorithme} à une autre classe de graphes que sont les \textbf{graphes séries-parallèles}. Une première partie du projet consistera en une étude théorique menant à la création d’un algorithme efficace de visualisation de graphes séries-parallèles. Pour ce faire, une compréhension de l’algorithme existant pour les arbres sera nécessaire, afin de l’adapter à ces nouveaux besoins. Une seconde partie, pratique, consistera à \textbf{implémenter cet algorithme} dans un langage au choix. L’implémentation permettra un \textbf{rendu des graphes sous plusieurs formats} (en utilisant graphviz, svg, ...) mais particulièrement sous forme de figure TikZ. TikZ est un ensemble de macro de dessin pour le langage Tex, langage dominant dans la rédaction et publication de contenu scientifique.
\newline
\newline
Les tâches à effectuer :
\begin{itemize}
    \item s’approprier l’usage du visualiseur d’arbres 
    \item adapter l’algorithme de visualisation afin d’afficher des graphes séries-parallèles 
    \item s’approprier l’usage de LateX et de sa bibliothèque tikz 
    \item implémenter l’algorithme (langage au choix) afin de générer le code tikz encodant la visualisation du graphe
    \item éventuellement enrichir les possibilités d’affichage
\end{itemize}






%%%%%%%%%%%%%%%%%%%%%%%%%%%%%%%%%%%%%%%%%%%%%%%%%%%%%%%%%%%%%%%%%%%%%%%%%%%%%%%%%%%%%%%%%%%%%%%%%%%%
\section{Etat des lieux}
\fcolorbox{black}{LightBlue2}{\parbox{\linewidth}{
TODO : Probablement une section à renommer.
\newline
}}
\subsection{Format DOT/Tikz + Graphviz}
\fcolorbox{black}{LightBlue2}{\parbox{\linewidth}{
TODO : 
\newline
Clairement à renommer.
\newline
Ici, nous introduisons le format DOT (que nous recevons en entrée) et le format Tikz (Que nous fournissons éventuellement en sortie, sachant que cette partie n'est pas encore implémentée).
\newline
Grossièrement, c'est juste pour que le lecteur puisse comprendre sur quoi nous nous basons.
\newline
Est-ce vraiment nécessaire ? Sachant que nous en parlons section 5 ?
\newline
Aborder la façon de faire de graphviz pour afficher les graphes (cf. tri topologique) ; ouverture sur deux travaux pour afficher des graphes séries-parallèles, d'une manière différente de la notre. Ne pas s'étendre sur ce dernier point, simplement renvoyer en annexe si le lecteur est intéressé.
}}
\subsection{Visualisation d'un arbre (from TreeDisplay)}
\fcolorbox{black}{LightBlue2}{\parbox{\linewidth}{
TODO : 
\newline
Dire qu'on s'est basé sur les recherches du TreeDisplay d'un pstl précédent
}}
\subsubsection{Algorithme linéaire}
\fcolorbox{black}{LightBlue2}{\parbox{\linewidth}{
TODO : 
\newline
Présentation du pseudo-code de l'algorithme (à vérifier les algos pseudo-codes, j'ai écrit ça il y a un moment !)
}}
\begin{algorithm}[H]
step : step between nodes\\
 \KwData{Node node ; int dept=0}
 \KwResult{void}
 node.y = depth*step\;
 \For{$Node$ $child$ : $node.getChildren()$}{
    $computeDown(child, depth+1)$\;
 }
 $int$ $nbChildren = node.getChildren().size()$\; 
 $int$ $place = 0$\;
\eIf{$nbChildren = 0$}{
   $place \leftarrow next[depth]$\;
   $node.x \leftarrow place$\;
}{
   $place \leftarrow (node.getChildren().get(0).x + node.getChildren().get(nbChildren-1).x) /2$\;
}
$offset[depth] = Math.max(offset[depth], next[depth]-place)$\;
\If{nbChildren != 0}{
    $node.x = place + offset[depth]$\;
}
$next[depth] = n.x+step$\;
$node.offset = offset[depth]$\;
 \caption{computeDown(Node, int) \textbf{[$\mathcal{O(N)}$]}}
\end{algorithm}

\begin{algorithm}[H]
step : step between nodes\\
 \KwData{Node node ; int offsum=0}
 \KwResult{void}
 $node.x = node.x + offsum$\;
 $ offsum += n.offset$\;
 \For{$Node$ $child$ : $node.getChildren()$}{
    $addOffsets(child, offsum)$\;
 }
 \caption{addOffsets(Node, int) \textbf{[$\mathcal{O(N)}$]}}
\end{algorithm}
\subsubsection{Explications et fonctionnement}
\fcolorbox{black}{LightBlue2}{\parbox{\linewidth}{
TODO : 
\newline
Explications détaillées mais concises de l'algorithme, pourquoi et comment ça fonctionne, pourquoi c'est linéaire.
\newline
En bref, l'important est surtout que le lecteur puisse \underline{comprendre} sur quoi nous nous sommes basés.
\newline
A vérifier l'explication, et de toute manière à compléter.
}}
Dans l'algorithme 1, on parcourt l'arbre de manière infixe afin de placer un noeud par rapport à ses fils, en fonction de la place disponible. Si la place voulue n'est pas disponible, on retient la différence entre la place voulue et la place reçue.\\
\\
C'est ici qu'est l'astuce :
Dans le même algorithme, on aurait re-parcouru le sous-arbre pour faire un décalage immédiat. Or si à chaque fois que l'on remonte un parent on doit décaler son sous-arbre, on pourrait le faire potentiellement \textbf{$\mathcal{O}(N^2)$} fois.\\
Pour éviter celà, on a juste retenu le décalage, et on va redescendre l'arbre une unique fois en sommant les décalages à la descente.\\





%%%%%%%%%%%%%%%%%%%%%%%%%%%%%%%%%%%%%%%%%%%%%%%%%%%%%%%%%%%%%%%%%%%%%%%%%%%%%%%%%%%%%%%%%%%%%%%%%%%%
\section{Les Graphes Séries-Parallèles (SPG)}
\subsection{Définitions}
\subsubsection{Généralités}
\fcolorbox{black}{LightBlue2}{\parbox{\linewidth}{
TODO :
\newline
Définition récursive des SPG qui nous a été donnée. (i.e. SPG est Ensemble vide ou noeud simple ; ou un SPG composé avec un autre SPG)
}}
\subsubsection{Compositions}
\fcolorbox{black}{LightBlue2}{\parbox{\linewidth}{
TODO : Les compositions possibles avec les séries parallèles (parallèle et série donc)
\newline
Sans pseudo-code, ça ne semble pas pertinent.
\newline
}}
\textbf{Composition en série}
\newline
\textbf{Composition en parallèle}
\subsection{Structure de données et représentation}
\fcolorbox{black}{LightBlue2}{\parbox{\linewidth}{
TODO :
\newline
Ici on aborde simplement notre manière de représenter un SPG en mémoire (Liste d'adjacence donc) ; pas de code ici !!
\newline
N'oublions pas de jolis dessins !
}}
\subsection{Reconnaissance}
\fcolorbox{black}{LightBlue2}{\parbox{\linewidth}{
TODO :
\newline
Comment reconnaitre un SPG, quels outils ?
}}
\subsubsection{Détection de cycle}
\fcolorbox{black}{LightBlue2}{\parbox{\linewidth}{
TODO :
\newline
Explication de l'utilité de détection de cycle (Pas de cycle dans un SPG) + pseudo-code + complexité (sans démonstration)
}}
\subsubsection{Composantes connexes}
\fcolorbox{black}{LightBlue2}{\parbox{\linewidth}{
TODO :
\newline
Explication de l'utilité des composantes connexes + pseudo-code + complexité (sans démonstration)
}}



%%%%%%%%%%%%%%%%%%%%%%%%%%%%%%%%%%%%%%%%%%%%%%%%%%%%%%%%%%%%%%%%%%%%%%%%%%%%%%%%%%%%%%%%%%%%%%%%%%%%
\section{Visualisation de Graphes Séries-Parallèles}
\subsection{Différences avec un arbre}
\fcolorbox{black}{LightBlue2}{\parbox{\linewidth}{
TODO :
\newline
Leur algo nous convient à moitié, explications des différences majeures avec les SPG
}}

\subsection{Contraintes d'affichage}
\fcolorbox{black}{LightBlue2}{\parbox{\linewidth}{
TODO :
\newline
Donc finalement, les contraintes d'affichage en plus et/ou en moins pour les SPG. Que doit-on faire.
\newline
Peut-être à merge avec la sous-section précédente ?
}}
Noeuds parents centrés vis-à-vis de leurs enfants, si ils en ont plusieurs. (abscisse)\\
Noeuds enfants centrés vis-à-vis de leurs parents, si ils en ont plusieurs. Ou par rapport à leur join associé (abscisse)\\
Les noeuds sont placés en fonction de leur profondeur.                      (ordonnée)\\
Les arcs ne doivent pas s'intersecter.\\
Un même sous-arbre est dessiné de la même manière, peu importe sa place.\\

\subsection{Algorithme}
\subsubsection{Etape 1}
\fcolorbox{black}{LightBlue2}{\parbox{\linewidth}{
TODO :
\newline
A renommer + pseudo-code et explications détaillés + complexité
}}
\subsubsection{Etape 2}
\fcolorbox{black}{LightBlue2}{\parbox{\linewidth}{
TODO :
\newline
A renommer + pseudo-code et explications détaillés + complexité
}}
\subsubsection{Etape 3}
\fcolorbox{black}{LightBlue2}{\parbox{\linewidth}{
TODO :
\newline
A renommer + pseudo-code et explications détaillés + complexité
}}
\subsubsection{Etape 4}
\fcolorbox{black}{LightBlue2}{\parbox{\linewidth}{
TODO :
\newline
A renommer + pseudo-code et explications détaillés + complexité + complexité totale de l'algo (sans démonstration, c'est assez clair)
}}

\section{Implémentation dans le cadre du PSTL}
\fcolorbox{black}{LightBlue2}{\parbox{\linewidth}{
TODO :
\newline
Ici on pourrait rajouter des métriques pour faire plaisir au jury... Typiquement le nombre de ligne de code même si..Ce n'est pas très pertinent ! :-)
}}
\subsection{Parseur}
\fcolorbox{black}{LightBlue2}{\parbox{\linewidth}{
TODO :
\newline
Parseur pour le format DOT,a-t-on le temps d'en faire un autre format rapidement ?
\newline
Explications de la lib utilisée pour construire notre graphe (lib graphviz)
}}
\subsection{Générateur}
\fcolorbox{black}{LightBlue2}{\parbox{\linewidth}{
TODO :
\newline
Génération d'image (PNG, JPEG, BMP)
\newline
Génération pour TIKZ, pas encore implémentée ; a-t-on le temps d'en faire d'autre ?
}}


\subsection{Interface Graphique}
\fcolorbox{black}{LightBlue2}{\parbox{\linewidth}{
TODO :
\newline
Présentation (très très) rapide du logiciel et de son interface d'affichage (sans oublier de préciser que le temps d'affichage par swing peut prendre un moment...)
}}





%%%%%%%%%%%%%%%%%%%%%%%%%%%%%%%%%%%%%%%%%%%%%%%%%%%%%%%%%%%%%%%%%%%%%%%%%%%%%%%%%%%%%%%%%%%%%%%%%%%%
\section{Conclusion}
\fcolorbox{black}{LightBlue2}{\parbox{\linewidth}{
TODO :
\newline
Un premier paragraphe pour la conclusion en elle-même, un second pour les ouvertures et suites envisageables.
}}


%%%%%%%%%%%%%%%%%%%%%%%%%%%%%%%%%%%%%%%%%%%%%%%%%%%%%%%%%%%%%%%%%%%%%%%%%%%%%%%%%%%%%%%%%%%%%%%%%%%%
\section{Annexe}
\subsection{Prise en main rapide}
\fcolorbox{black}{LightBlue2}{\parbox{\linewidth}{
TODO :
\newline
Prise en main rapide de notre logiciel, ne devrait pas être trop long.
Peut-être des choses à rajouter dans les annexes ? Pas à première vue..
}}

\subsection{Bibliographie}
\fcolorbox{black}{LightBlue2}{\parbox{\linewidth}{
TODO :
\newline
Ne pas oublier la biblio ! Et le lien vers le github
}}

\end{document}